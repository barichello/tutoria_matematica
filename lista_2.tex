\documentclass[main_estudante.tex]{subfiles}

\begin{document}

%\end{comment}
\paraAmbos

\chapter{Revisão para integrais trigonométricas}

%\end{comment}
\paraAlunos

\section{Apresentação}

Este capítulo usará a integral de uma função aparentemente simples, $f(x)=\sqrt{4-x^2}$, como mote para revisar alguns conteúdos utilizados em duas técnicas de integração ligadas a funções trigonométricas que você irá estudar em MA111.

Em linhas gerais, o que queremos reforçar neste capítulo são algumas igualdades envolvendo senos, cossenos e tangentes que nos permitirão fazer mudanças estratégicas em certas funções de modo a torná-las mais simples de integrar.

%\end{comment}
\paraTutores

\section{Comentários iniciais}

Esse capítulo é focado em algumas identidades trigonométricas que são fundamentais para as técnicas de integração chamadas \textbf{integrais trigonométricas} e \textbf{subsituição trigonométricas}.

Além de relembrar essas identidades, este capítulo foca em usos compatíveis com o que será feito na disciplina de Cálculo Diferencial e Integral.

Ao longo do capítulo algumas integrais envolvendo as funções trigonométricas seno, cosseno e tangente serão necessárias.

Um ponto importante em relação às identidades trigonométricas é que os estudantes vejam que todas podem ser obtidas a partir de um conjunto menor formado, essencialmente por:

$$\sin^2(x)+\cos^2(x)=1$$

$$\sin(a+b)=\sin(a)\cos(b)+\sin(b)\cos(a)$$

$$\cos(a+b)=\cos(a)\cos(b)-\sin(a)\sin(b) $$

A primeira pode ser rapidamente obtida a partir do teorema de pitágoras, mas as outras duas são mais trabalhosas para obter e precisam ser decoradas. Além de lembrar das identidades acima, é importante saber como usá-la e esperamos que a sequência de questões propostas faça esse trabalho.

%\end{comment}
\paraAmbos

\section{Quando usar}

Este capítulo deve ser resolvido antes do professor de MA111 abordar substituição trigonométrica e integrais trigonométricas, ou assim que ele iniciar esses tópicos.

\section{Conteúdos anteriores}

O conteúdo deste capítulo gira em torno das funções trigonométricas com que você vem trabalhando na disciplina MA111. Como o nosso foco aqui será menos no cálculo das integrais e mais nas transformações que podemos fazer em alguns integrandos, os pré-requisitos deste capítulo são:
\begin{itemize}
 \item Definição de seno e cosseno;
 \item Definição de tangente e secante.
 \item Integral e derivada das funções seno e cosseno.
\end{itemize}

Esses tópicos não serão cobertos durante as atividades de tutoria. Se você acha que não sabe o suficiente sobre algum deles, sugerimos que volte ao capítulo sobre polinômios deste material e os revise.

\newpage

\section{Questões}

Como dito na introdução, vamos explorar a integral da função real $f(x)=\sqrt{4-x^2}$. Antes de partirmos para a sua integral de fato, vamos explorar alguns aspectos da função.

\subsection*{Domínio, imagem e gráfico}

Vamos começar com algumas propriedades fundamentais dessa função.

\notaTutor{As duas primeiras questões bastante introdutórias e tem como objetivo estabelecer bem o cenário que servirá para este capítulo. Na verdade, o capítulo todo girará em torno da resolução (e entendimento) da integral de $f(x)$.}

\begin{questao}
Considerando a função $f(x)=\sqrt{4-x^2}$, responda:
\begin{enumerate}[a)]
\item Qual é o seu domínio?
\item Quais são as raízes da função?
\item Em que ponto o seu gráfico corta o eixo $Y$?
\item Qual é a imagem da função?
\end{enumerate}
\end{questao}

\begin{gabarito}
	\begin{gabaritoQuestao}
		a) $[-2;2]$, b) $-2$ e $2$, c) $(0;2)$, d) $[0;2]$.
	\end{gabaritoQuestao}
\end{gabarito}

Como você deve ter concluido nas questões acima, essa função é definida apenas no intervalo $[-2;2]$. Fora dele, teríamos a raiz de um número negativo, o que não é definido nos números reais.

No que diz respeito à imagem da função, é importante salientar que a expressão $f(x)=\sqrt{4-x^2}$ poderia ser interpretada como ambígua, já que não esclarece se deve ser tomada a raiz positiva ou negativa de $4-x^2$ e uma função deve assumir um único valor para cada valor de $x$. Entretanto, convencionalmente, o símbolo de raiz quadrada no contexto de funções deve ser interpretado como a raiz quadrada positiva. Nesse caso, a imagem da função é o intervalo $[0;2]$

\begin{questao}
Manipule algebricamente a função na forma $y=\sqrt{4-x^2}$ de modo a obter uma expressão que seja familiar. 
\begin{enumerate}[a)]
\item Que curva essa expressão representa?
\item Esboce o gŕafico da função $f(x)$.
\end{enumerate}
\end{questao}

\begin{gabarito}
	\begin{gabaritoQuestao}
		a) circunferência de raio 2 e centro na origem.
	\end{gabaritoQuestao}
\end{gabarito}

\subsection*{Significado da integral}

\notaTutor{Ainda em aspectos introdutórios, essa questão tenta salientar uma possível conexão entre a integral pedida e funções trigonométricas.}

\begin{questao}
Vamos considerar agora a integral $\int_{-2}^{2} f(x)dx$.
\begin{enumerate}[a)]
\item Qual é o significado geométrico dessa integral?
\item Com base na resposta anterior, qual deve ser o valor da integral (sem efetuar a integração de fato)?
\end{enumerate}
\end{questao}

\begin{gabarito}
	\begin{gabaritoQuestao}
		a) a integral é igual à área de uma semi-circunferência de raio 2, b) $2\pi$.
	\end{gabaritoQuestao}
\end{gabarito}

Por se tratar da área de um semi-círculo de raio 2, sabemos que a integral de $f(x)$ entre $-2$ e $2$ deve ser igual a $\frac{1}{2} \cdot \pi 2^2 = 2\pi$. Em princípio, esse valor deve parecer estranho: como obteríamos $\pi$ a partir de uma expressão algébrica que envolve apenas somas, multiplicações e raízes? Essa conexão é o que está por trás da técnica de integração chamada substituição trigonométrica e esse é o tópico central deste capítulo.

\subsection*{Transformando a integral, tentativa 1}

Apesar de parecer relativamente simples, a integral $\int \sqrt{4-x^2}dx$ não é nem um pouco simples de resolver através de trocas de variáveis.

\notaTutor{Essa questão busca convencer o estudante de que a integral pedida não pode ser resolvida através de uma simples troca de variável. Não deixe que eles gastem muito tempo com ela.}

\begin{questao}
Para que você sinta a dificuldade, tente obter um integrando mais simples usando a troca de variáveis $u=4-x^2$. Qual é a nova integral obtida?
\end{questao}

\begin{gabarito}
	\begin{gabaritoQuestao}
		$-2\int\sqrt{4u-u^2}du$
	\end{gabaritoQuestao}
\end{gabarito}

Você pode tentar outras trocas de variáveis, mas provavelmente você vai chegar em situações muito próximas à que você obteve acima: alguma raiz quadrada acaba reaparecendo, o que dificulta a obtenção de uma primitiva para a função.

\subsection*{Transformando a integral, tentativa 2}

O que faremos então é uma grande mudança na abordagem. Primeiro, note que $\sqrt{4-x^2}$ se parece com expressões que ocorrem ao resolvermos questões envolvendo teorema de pitágoras.

\notaTutor{Com essa questão introduzimos um argumento geométrico para a troca de variáveis que será feita e que fundamenta a técnica de integração chamada de substituição trigonométrica. O argumento é bastante simples, mas muito poderoso e pode ser estendido na direção de coordenadas polares, mas isso não será feito aqui.}

\begin{questao}
Considerando o triângulo retângulo abaixo:
\begin{enumerate}[a)]
\item Use o teorema de pitágoras para obter uma expressão para o comprimento do cateto vertical.
\item Use as relações trigonométricas acerca do ângulo $\theta$ para obter uma outra expressão para o comprimento do cateto vertical.
\item Use as relações trigonométricas acerca do ângulo $\theta$ para obter uma outra expressão para o comprimento do cateto $x$.
\end{enumerate}
\end{questao}

\begin{figure}[h]
\centering
\includegraphics[width=0.25\textwidth]{./img/l2q4.png}
\end{figure}

\begin{gabarito}
	\begin{gabaritoQuestao}
		a) $\sqrt{4-x^2}$, b) $2\cos(\theta)$, c) $2\sin(\theta)$.
	\end{gabaritoQuestao}
\end{gabarito}

Note que o cateto vertical pode ser expresso de duas maneiras, como a raiz da função dada inicialmente ou em termos do $\sin(\theta)$. Como sabemos como integrar senos, vamos tentar usar essa ideia para uma troca de variáveis que nos ajude de fato.

\begin{questao}
Rescreva a integral $\int_{-2}^{2} \sqrt{4-x^2}dx$ usando a troca de variáveis $\sqrt{4-x^2}=\cos(\theta)$ (lembre-se que $x=2\sin(\theta)$).
\end{questao}

\notaTutor{Finalmente foi realizada a troca de variáveis. Apesar de o texto salientar que o diferencial e os limites da integral devem ser ajustados, certifique-se de que os estudantes estão dando atenção a esses pontos. Se eles estiverem com muitas dificuldades nesse momento é prvável que estejam com dificuldades em integrais como um todo. Nesse caso, sugira que calculem as integrais indefinidas $\int e^{2x+3}dx$ e $\int x.\sin(x^2)dx$. Se os estudantes não conseguirem resolvê-las, eles estão com dificuldades mais básicas sobre a regra da substituição e você pode recomendar a leitura dos exemplos da seção 5.5 do livro \sugestao{Calculus}. Se eles conseguirem, provavelmente vale a pena ajudá-los passo-a-passo com a troca de variáveis desse caso especificamente.}

\begin{gabarito}
	\begin{gabaritoQuestao}
		$4 \int_{-\pi/2}^{\pi/2} \cos^2(\theta)d\theta$.
	\end{gabaritoQuestao}
\end{gabarito}

Não se esqueça de ajustar os limites da integral. Como $x=2\sin(\theta)$, e $x$ deve variar de $-2$ a $2$ a opção mais simples é variarmos $\theta$ de $-\pi/2$ a $\pi/2$. Portanto, nossa nova integral é:

$$\int_{-\pi/2}^{\pi/2} 2\cos(\theta). 2\cos(\theta)d\theta = 4 \int_{-\pi/2}^{\pi/2} \cos^2(\theta)d\theta$$

O objetivo de obter uma integral que não envolve-se raízes foi atingido!

\subsection*{Pare e pense}

A transformação feita acima foi bem sucedida por um motivo: ao fazermos a troca $x=2\sin(\theta)$ também pudemos trocar a expressão $\sqrt{4-x^2}$ por $\cos(\theta)$. Vimos essa troca no triângulo, mas ela pode ser vista algebricamente graças à igualdade trigonométrica fundamental $\sin^2(\alpha) + \cos^2(\alpha)=1$. Ela implica que $\sin^2(\alpha)=1-\cos^2(\alpha)$. Portanto:

$$\sqrt{4-x^2} = \sqrt{4-4\sin^2(\theta)} = \sqrt{4-4(1-\cos^2(\theta))} = \sqrt{\cos^2(\theta)} = \cos(\theta)$$

Essa visão algébrica do processo é útil para que você possa aplicar essa troca em mais situações, não apenas naquelas que possam ser interpretadas em um triângulo retângulo. Por isso, o quadro abaixo traz as três variações da igualdade trigonométrica fundamental que lhe podem ser úteis.

\begin{shaded*}
$$\sin^2(\alpha) + \cos^2(\alpha)=1$$
$$\tan^2(\alpha) + 1= \sec^2(\alpha)$$
\end{shaded*}

Note que, na verdade, a segunda igualdade pode ser obtidas a partir da primeira, basta dividir todos os termos dela por $\cos^2(\alpha)$.

\begin{questao}
Considerando as igualdades acima:
\begin{enumerate}[a)]
\item Qual troca de variável você utilizaria se a função a ser integrada fosse $g(x)=\sqrt{1+x^2}$?
\item Como ficaria a integral $\int \sqrt{1+x^2} dx$ quando essa troca for completada?
\end{enumerate}
\end{questao}

\notaTutor{Essa questão foi incluída para chamar a atenção para o impacto que o sinal do termo dentro da raiz tem sobre a escolha da troca a ser usada. e para mostrar que a técnica pode ser aplicada em mais alguns casos.}

\begin{gabarito}
	\begin{gabaritoQuestao}
		a) $x=\tan(\alpha)$ e, portanto, $\sqrt(4+x^2)=\sec(\alpha)$, b) $\int \sec^3(\alpha)d\alpha$.
	\end{gabaritoQuestao}
\end{gabarito}

\subsection*{Voltando à integral}

Antes da seção anterior, tínhamos chegado em $4\int_{-\pi/2}^{\pi/2} cos^2(\theta)d\theta$. Porém, mais uma vez, apesar da integral de $cos(\theta)$ ser simples, a de $cos^2(\theta)$ não é.

Se você tentar fazer a troca de variáveis $u=cos(\theta)$, vai notar que um seno aparecerá na expressão quando você transformar $d\theta$ em $du$. Esse problema é bastante comum quando integramos potências de seno e cosseno. O que pode ser feito para contornar essa dificuldade é utilizar uma outra família de igualdades trigonométricas. Talvez você as tenha estudado no Ensino Médio, caso não, sugerimos que memorize as duas igualdades mostradas abaixo, pois serão muito usadas nas listas de MA111.

\begin{shaded*}
$$\cos(2\alpha) = 2\cos^2(\alpha)-1$$
$$\sin(2\alpha) = 2\sin(\alpha).\cos(\alpha)$$
\end{shaded*}

Note que a primeira delas permite transformar um cosseno ao quadrado em um cosseno, enquanto que a segunda permite transformar um produto de seno por cosseno em um seno apenas. Essas igualdades são úteis para mudarmos o formato de potências de funções trigonométricas para formatos mais simples de integrar, como faremos agora.

Antes de retomarmos a integral, resolva a seguinte questão sobre as igualdades acima para que você se familiarize com o uso que será feito delas em MA111.

\notaTutor{Essa questão explora as igualdades frequentemente utilizadas na técnica de integração chamada de integrais trigonométricas e que é normalmente aplica a potências de senos e cossenos. Como ressaltado no texto para o aluno, as igualdades permitem a redução da potência em senos e cossenos e ambas derivam das fórmulas de $\sin(a+b)$ e $\cos(a+b)$.

O último item da questão exige o uso da identidade duas vezes consecutivas. Apesar de destoar das demais questões propostas devido à quantidade de manipulações algébricas, insista que os estudantes o resolvam por inteiro pois isso será esperado deles em MA111.}

\begin{questao}
Considerando as igualdades acima:
\begin{enumerate}[a)]
\item Isole $\cos^2(\alpha)$ na primeira igualdade.
\item Use a igualdade $\sin^2(a) + \cos^2(a)=1$ e a resposta do item anterior para obter uma expressão que transforma $\sin^2(a)$ em $\cos(2a)$.
\item Rescreva a expressão $\sin^4(a)$ como uma expressão composta por senos e cossenos com as menores potências possíveis.
\end{enumerate}
\end{questao}

\begin{gabarito}
	\begin{gabaritoQuestao}
		a) $\cos^2(\alpha)=\frac{\cos(2\alpha)+1}{2}$, b) $\sin^2(\alpha)=\frac{1-\cos(2\alpha)}{2}$, c) $\sin^4(\alpha)=\frac{1}{8} \cdot (3-4\cos(2\alpha)+\cos(4\alpha))$.
	\end{gabaritoQuestao}
\end{gabarito}

Note que a expressão obtida no item c acima pode parecer complicada, mas não envolve nenhuma potência e seria simples de integrar, ou seja, você fez quase todo o trabalho necessário para calcular $\int \sin^4(x)dx$.

Agora, voltemos à nossa integral.

\begin{questao}
Rescreva $4\int_{-\pi/2}^{\pi/2} cos^2(\theta)d\theta$, usando as igualdades anteriores, de modo que não haja mais potências no integrando.
\end{questao}

\begin{gabarito}
	\begin{gabaritoQuestao}
		$2\int_{-\pi/2}^{\pi/2} (\cos(2\theta)+1)d\theta$
	\end{gabaritoQuestao}
\end{gabarito}

O que você fez na questão acima não é uma mudança de variável, mas sim uma manipulação algébrica que resultou em uma nova expressão equivalente à que tínhamos antes. Note que a expressão que você acabou de obter não tem mais cosseno ao quadrado e é mais simples de integrar.

\notaTutor{Após a sequência de passos seguida nas questões anteriores, a integral final é relativamente simples e o resultado é esperado: $2\pi$. Fique atento a pequenas variações na resposta que podem advir de erros procedimentais, como a omissão de algum sinal ou de perda de alguma constante.}

\begin{questao}
Finalize essa questão calculando o valor da integral definida obtida na questão anterior.
\end{questao}

Você deve ter obtido $2\pi$ como resposta, assim como havíamos previsto no início do capítulo.

\section{Mensagem final}

O grande objetivo deste capítulo era relembrar algumas identidades trigonométricas que são usadas com frequência quando utilizamos as técnicas de integração chamadas substituição trigonométrica e integrais trigonométricas.

A primeira família de igualdades (que derivam de $\sin^2(\alpha)+\cos^2(alpha)=1$) nos permitem:
\begin{enumerate}[1)]
\item Converter senos em cossenos, ou vice-versa;
\item Lidar com expressões do tipo $a^2-x^2$ fazendo a troca $x=\sin(\alpha)$;
\item Lidar com expressões do tipo $a^2+x^2$ fazendo a troca $x=\tan(\alpha)$.
\end{enumerate}

A segunda família de igualdades, envolvendo $\sin(2\alpha)$ e $\cos(2\alpha)$, nos permite:
\begin{enumerate}[1)]
\item Reduzir potências de funções trigonométricas para potências menores ($\cos^2(alpha)$ em $\cos(2\alpha)$), por exemplo;
\item Converter expressões com produtos de senos e cossenos em expressões envolvendo $\sin(2\alpha)$.
\end{enumerate}

Sugerimos agora que você leia os exemplos 1, 2 e 3 da seção 7.2 do livro \sugestao{Calculus} e os exemplos 1, 2, 3 e 4 da seção 7.3. Em seguida, parta para a lista de exercícios da disciplina.

%\end{comment}
\paraTutores

\section{Comentários finais}

Durante as atividades da tutoria não se preocupe em identificar e listar todos os casos em que as duas técnicas de integração em questão costumam ser sub-divididas. Isso será feito pelo professor de MA111. Por outro lado, é importante que os estudantes consigam lembrar das identidades trigonométricas utilizadas ao longo do capítulo. Insista na importância desse aspecto!

Após resolverem as questões propostas, sugerimos aos estudantes que leiam alguns exemplos das seções 7.2 e 7.3 do livro \sugestao{Calculus} e então partam para a lista de exercícios da disciplina.

\newpage

%\end{comment}
\paraAmbos

\section{Gabarito}

Confira as respostas para as questões e \textbf{não se esqueça de registrar o seu progresso}.

\imprimeGabarito

%\end{comment}
\paraAmbos

\end{document}
