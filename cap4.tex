\documentclass[main.tex]{subfiles}

\begin{document}

%\end{comment}
\paraAmbos

\chapter{Trigonometria e Vetores}

%\end{comment}
\paraAlunos

\section{Apresentação}

Além de ser uma disciplina em si mesma, a Geometria Analítica inaugura um jeito de encarar problemas geométricos que será o mais comum durante disciplinas matemáticas no ensino superior. A Geometria Analítica nasce da junção de técnicas da Álgebra (equações, funções) e da Geometria Plana (medidas, ângulos, semelhança) graças ao uso de um sistema de eixos cartesianos. Mas além de permitir o uso de elementos desses dois universos, essa junção abre portas para a criação de novos elementos, como os vetores.

O que faremos neste capítulo é revisitar alguns conceitos centrais da trigonometria que você deve ter estudado no Ensino Médio, mas agora enfatizaremos os usos que serão feitos na disciplina de Geometria Analítica. Além disso, utilizaremos constantemente eixos cartesianos em questões que poderiam ser formuladas apenas com elementos geométricos e utilizaremos vetores para formular e resolver algumas das questões.

\section{Pré-requisitos e Auto-avaliação inicial}

Os pré-requisitos para este capítulo são:
\begin{itemize}
 \item Noções básicas de trigonometria no triângulo retângulo (seno e cosseno);
 \item Noções básicas sobre plano cartesiano (coordenadas).
\end{itemize}

Esses tópicos não serão cobertos durante as atividades de tutoria. Se você acha que não sabe o suficiente sobre algum deles, sugerimos que se procure material de apoio antes de começar a resolver as questões desse capítulo.

Antes de começar, indique o quanto você acha que sabe sobre os seguintes itens:

\begin{center}
 \begin{tabular}{|p{25mm}||p{10mm}|p{10mm}|p{10mm}|p{10mm}|} 
 \hline
   & Nada & Muito pouco & Noções gerais & Muito\\
 \hline
 Calcular seno e cosseno em um triângulo retângulo &  &  &  &  \\ 
 \hline
 Obter um ângulo sabendo o seu seno ou cosseno &  &  &  &  \\
 \hline
 Usar seno e cosseno em uma calculadora &  &  &  &  \\
 \hline
 Representar vetores &  &  &  &  \\
 \hline
\end{tabular}
\end{center}

\section{Questões diagnósticas}

\begin{diagnostico}
Considerando a figura abaixo, calcule:
\begin{enumerate}[a)]
  \item A medida do segmento $\overline{AB}$.
  \item Seja $D$ a intersecção do lado $CB$ com a altura do triângulo relativa ao vértice $A$, calcule o comprimento de $\overline{BD}$.
\end{enumerate}
\end{diagnostico}

%\end{comment}
\paraTutores

\subsection{Gabarito}

a) $3\sqrt{3}$, b) $9/2$.

%\end{comment}
\paraAmbos
 
\begin{figure}[h]
\centering
\includegraphics[width=0.4\textwidth]{./img/c4d1.png}
\end{figure}

\begin{diagnostico}
Qual é a medida em graus do ângulo formado entre o eixo $X$ e o segmento de reta que liga a origem ao ponto $(3;1)$? \textit{Você pode usar a calculadora se julgar necessário.}
\end{diagnostico}

%\end{comment}
\paraTutores

\subsection{Gabarito}

a) aproximadamente $18,4\degree$.

\section{Quadro de orientação}

\begin{center}
 \begin{tabular}{|c c c |c|} 
 \hline
 1A e 1B & 2 & Onde começar\\
 \hline
 C & E & Questão 2 \\ 
 \hline
 C & C & Questão 5 \\ 
 \hline
\end{tabular}
\end{center}

%\end{comment}
\paraAmbos

\newpage

\section{Questões}

%\end{comment}
\paraTutores

\subsection{Comentários iniciais}

Neste capítulo, serão trabalhados conteúdos relacionados a trigonometria e vetores. A intenção é focar em aspectos da trigonometria que estejam mais próximos do que será usado em Geometria Analítica, porém, limitados ao contexto do plano cartesiano (duas dimensões).

As questões propostas não serão arranjos complicados de segmentos e círculos nos quais propriedades geométricas devem ser identificadas. No geral, a interpretação geométrica das questões será simples e a atenção será focada no uso que se faz das relações seno e cosseno e na interpretação vetorial dessas relações.

Vetores serão utilizados na formulação de algumas questões ao invés de pontos e segmentos, que são mais comuns ao longo do Ensino Médio. A intenção é familiarizar o estudante com esse conceito sem explorar suas propriedades específicas.

As funções inversas, arcosseno e arcocosseno, também serão utilizadas com bastante frequência, mas a ênfase não será no comportamento delas como função, mas no seu uso para resolução e interretação de problemas de geometria plana.

Este capítulo difere dos demais deste material em uma outra característica. Livros de Geometria Analítica raramente lidam com objetos bidimensionais pois nesse contexto as generalizações por trás de conceitos como produto escalar e vetorial são pouquíssimo úteis (e relações trigonométricas bastam). Por isso, não foi possivel indicar referências externas ao longo do capítulo e usar uma questão de fato extraída de um livro-texto ao final. Enquanto isso certamente representa uma perda em termos da proposta do material , esperamos construir ao longo dele um entendimento conceitual mais robusto em duas dimensões de conceitos que serão estudados exaustivamente em mais dimensões em Geometria Analítica e Álgebra Linear.

\subsection{Início do conteúdo para o aluno}

%\end{comment}
\paraAmbos

Lembre-se de checar com seu tutor em qual questão você deve começar.

\subsection*{Seno e cosseno}

Em Geometria Plana, definimos seno e cosseno de um ângulo $\alpha$ como sendo iguais a:

\begin{caixaExemplo}
$$ \sin(\alpha) = \frac{\text{cateto oposto}}{\text{hipotenusa}} \text{ e } \cos(\alpha) = \frac{\text{cateto adjacente}}{\text{hipotenusa}}$$
\end{caixaExemplo}

O valor do seno e do cosseno dos ângulos 0\degree, 30\degree, 45\degree, 60\degree e 90\degree, chamados de ângulos notáveis, são simples e recorrentes em problemas de geometria plana, portanto, espera-se que estudantes os tenham memorizados.

\begin{caixaExemplo}
\begin{center}
 \begin{tabular}{|c| c c c c c |} 
 \hline
  & 0\degree & 30\degree & 45\degree & 60\degree & 90\degree\\
 \hline
  seno & $0$ & $\frac{1}{2}$ & $\frac{\sqrt{2}}{2}$ & $\frac{\sqrt{3}}{2}$ & $1$\\ 
 \hline
 cosseno & $1$ & $\frac{\sqrt{3}}{2}$ & $\frac{\sqrt{2}}{2}$ & $\frac{1}{2}$ & $0$\\
 \hline
\end{tabular}
\end{center}
\end{caixaExemplo}

\begin{questao}
\notaTutor{O objetivo desta questão é iniciar o uso das relações trigonométricas seno e cosseno em um problema relativamente simples de geometria plana. O terceiro item é o único que não pode ser resolvido diretamente via aplicação de uma relação trigonométrica, ele exige o cálculo de $AC$ e $DC$. Além disso, a resposta final depende de uma subtração envolvendo raízes. Esteja atento para o caso de seus estudantes estarem co dificuldades nessa manipulação algébrica ao invés dos conceitos trigonométricos.

Se você achar que os estudantes tiveram dificuldade nessa questão, proponha variações em torno da mesma configuração dando como valores iniciais os dois ângulos (note que o mais externo deve ser menor do que o interno e por enquanto é desejável usar apenas ângulos notáveis) e o comprimento de um dos vários segmentos disponíveis. Como o objetivo é reforçar as relações trigonométricas e não praticar problemas de geometria, a repetição da configuração não é um problema.} Sabendo que $\overline{AB}=6$ e $B \hat{C} D$ na figura abaixo, calcule o comprimento dos segmentos:
\begin{enumerate}[a)]
\item $\overline{BC}$
\item $\overline{BD}$
\item $\overline{AD}$
\end{enumerate}
\end{questao}

\begin{gabarito}
	\begin{gabaritoQuestao}
		a) $3$, b) $2\sqrt{3}$, c) $2\sqrt{3}$.
	\end{gabaritoQuestao}
\end{gabarito}

\begin{figure}[h]
\centering
\includegraphics[width=0.5\textwidth]{./img/c4q1.png}
\end{figure}

\subsection*{Seno e cosseno de ângulos quaisquer}

Os valores listados na tabela acima são importantes, mas ângulos podem ocorrer em quaisquer medidas. Nesses casos, você pode usar a sua calculadora (a menos que o valor exato ou alguma aproximação específica seja fornecida). Na maioria dos celulares os botões cos e sin ficam disponíveis apenas no modo científico (acessível ao colocar o celular na horizontal) e é fundamental checar se o ângulo deve ser dado em graus (DEG) ou radianos (RAD).

\begin{questao}
\notaTutor{A intenção dessa questão é apenas ter certeza de que os estudantes sabem usar a calculadora dos seus celulares para obter o seno e o cosseno de ângulos dados em graus e em radianos. Além disso, insista que façam os arredondamentos corretamente pois isso será usado ao longo deste capítulo todo.

O último item é intencionalmente dúbio: a falta do $\pi$ induz a leitura do valor sendo em graus, mas note que o símbolo \degree não está presente, portanto, trata-se sim de uma medida em radianos. A resposta obtida em cada caso é bastante diferente (o correto é um valor próximo de 0 pois $1,5$ é um pouco menor que $\pi/2$, ou seja, $90\degree$.} Use o seu celular pra obter:
\begin{enumerate}[a)]
\item $\sin(\frac{4\pi}{9})$, arredondado para 2 casas decimais.
\item $\cos(\frac{\pi}{12})$, arredondado para 2 casas decimais.
\item $\sin(18\degree)$, arredondado para 1 casa decimal.
\item $\cos(22.5\degree)$, arredondado para 1 casa decimal.
\item $\cos(1,5)$, arredondado para 2 casas decimais.
\end{enumerate}
\end{questao}

\begin{gabarito}
	\begin{gabaritoQuestao}
		a) $0,98$, b) $0,97$, c) $0,3$, d)$0,9$, e) $0,07$.
	\end{gabaritoQuestao}
\end{gabarito}

Deste ponto em diante, use a calculadora sempre que os ângulos envolvidos não sejam notáveis e quando nenhuma aproximação for fornecida. Caso nenhum arredondamento específico seja pedido, use sempre duas casas decimais.

\subsection*{Geometria com ângulos quaisquer}

\begin{questao}
\notaTutor{A estrutura geométrica da questão não deve ser o foco da atenção dos estudantes, pois o objetivo desta questão é que usem a calculadora para obter o seno e cosseno de ângulos não notáveis. Propriedades básicas como a soma dos ângulos internos de um triângulo e o fato de o ângulo $\hat{C}$ não ser dividido ao meio pela altura podem emergir.

Assim como na primeira questão, você pode gerar variações dessa questão de maneira rápida: mesmo diagrama, dois ângulos dados e uma medida.} Sabendo que na figura abaixo $\overline{AC}=4$ obtenha a medida dos segmentos pedidos abaixo. Use uma calculadora se necessário.
\begin{enumerate}[a)]
\item A altura do triângulo relativa ao lado $\overline{AB}$.
\item $\overline{CB}$
\end{enumerate}
\end{questao}

\begin{gabarito}
	\begin{gabaritoQuestao}
		a) $3,76$, b) $9,24$.
	\end{gabaritoQuestao}
\end{gabarito}


\begin{figure}[h]
\centering
\includegraphics[width=0.5\textwidth]{./img/c4q3.png}
\end{figure}

\subsection*{Obtendo o ângulo a partir do seno ou cosseno}

Se você souber que um ângulo de um triângulo retângulo tem seno igual a $\frac{1}{2}$ você sabe que se trata de um ângulo de 30\degree, mas isso só é possível porque este valor está entre os poucos que fazem parte da tabela de valores notáveis. A pergunta então é como você pode saber a medida em graus ou radianos de um ângulo cujo seno é igual a $0,7$, por exemplo?

A resposta está nas funções \textbf{arcosseno} (arcsin, em inglês) e \textbf{arcocosseno} (arccos, em inglês). Por serem as funções inversas do seno e do cosseno, também são chamadas algumas vezes de $\sin^{-1}$ e $\cos^{-1}$. Tente encontrar essas funções na calculadora do seu celular (além de acessar o modo científico, em algumas calculadoras é necessário ativar o modo de funções inversas - INV) e calcular $\sin^{-1}(0.5)$. O resultado deve ser aproximadamente $0.5236$ (que é igual a $\pi/6$) ou $30\degree$.

\begin{questao}
\notaTutor{As funções arcosseno e arcocosseno são introduzidas aqui com o intuito de expandir as possibilidades de trabalho com ângulos e relações trigonométricas para além dos valores notáveis.} Use as funções $\sin^{-1}$ e $\cos^{-1}$ da sua calculadora para obter:
\begin{enumerate}[a)]
\item A medida em radianos, com duas casas decimais, do ângulo cujo seno é igual a $0,7$. Dica: procure pela opção DEG e RAD na calculadora para alternar entre graus e radianos.
\item A medida em radianos, com duas casas decimais, do ângulo cujo cosseno é igual a $0,4$.
\item A medida em graus, com zero casas decimais, do ângulo cujo seno é igual a $0,6$. 
\item A medida em graus, com zero casas decimais, do ângulo cujo cosseno é igual a $0,87$.
\end{enumerate}
\end{questao}

\begin{gabarito}
	\begin{gabaritoQuestao}
		a) $0,78$, b) $1,16$, c) $37\degree$, d) $30\degree$.
	\end{gabaritoQuestao}
\end{gabarito}

\subsection*{Ângulos e eixos cartesianos}

\begin{questao}
Considerando a imagem abaixo, calcule:
\begin{enumerate}[a)]
\item O comprimento do segmento $\overline{OP}$. \textit{$O$ se refere à origem, ou seja, ao ponto $(0;0)$}.
\item O seno e o cosseno do ângulo determinado pelo segmento $\overline{OP}$ e o eixo $X$. 
\item A medida em radianos e em graus desse ângulo.
\end{enumerate}
\end{questao}

\begin{gabarito}
	\begin{gabaritoQuestao}
		a) $\sqrt{13}$, b) o seno vale $2\sqrt{13}/13$ e o cosseno $3\sqrt{13}/13$, c) $0,59$ e $34\degree$.
	\end{gabaritoQuestao}
\end{gabarito}

\begin{figure}[h]
\centering
\includegraphics[width=0.5\textwidth]{./img/c4q5.png}
\end{figure}

\subsection*{Vetores}

Na questão anterior, mais do que o ponto $P$, o elemento chave foi o segmento $\overline{OP}$. Em Geometria Analítica um dos conceitos mais importantes é o de \textbf{vetor}. Um vetor pode ser pensado como um segmento com direção, tipicamente uma seta. No caso da questão anterior, poderíamos falar do vetor $\overrightarrow{OP}$, ou simplesmente $\overrightarrow{P}$ uma vez que a outra extremidade é a origem.

Todo vetor possui três propriedades que o definem: a \textbf{norma} (nome dado ao comprimento se estivéssemos pensando em um segmento), a \textbf{direção} (dada pelo ângulo entre o vetor e a parte positiva do eixo X no sentido anti-horário) e o \textbf{sentido} (no caso do nosso exemplo, o sentido é de $O$ a $P$ e não de $P$ a $O$).

Pode parecer artificial usar esse conceito já que ele pode ser descrito em termos dos pontos das extremidades do vetor ou do segmento conectando esses dois pontos, mas o conceito de vetor abre novas possibilidades bastante poderosas em matemática pura, que serão estudadas em disciplinas como Geometria Analítica e Álgebra Linear, e em diversas aplicações, em Física e engenharias em geral.

Não estudaremos essas potencialidades dos vetores neste capítulo, mas utilizaremos o conceito em algumas questões para que você se familiarize com ele gradualmente.

\begin{questao}
\notaTutor{Essas duas questões servem como fixação do uso das funções arcosseno e arcocosseno bem como para introduzir o eixo cartesiano e o conceito de vetor. Não se preocupe com tecnalidades do conceito de vetor neste ponto (isso é conteúdo para Geometria Analítica).
} Considere os dois vetores representados abaixo.
\begin{enumerate}[a)]
\item Calcule a norma de $\overrightarrow{V}$ e $\overrightarrow{U}$.
\item Obtenha, em graus, os ângulos determinados por $\overrightarrow{V}$ e $\overrightarrow{U}$.
\item Determine, em graus, a medida do ângulo entre $\overrightarrow{V}$ e $\overrightarrow{U}$.
\end{enumerate}
\end{questao}

\begin{gabarito}
	\begin{gabaritoQuestao}
		a) $\Arrowvert V \Arrowvert = \sqrt{5}$ e $\Arrowvert U \Arrowvert = \sqrt{10}$, b) $63\degree$ e $18\degree$, c) $45\degree$.
	\end{gabaritoQuestao}
\end{gabarito}


\begin{figure}[h]
\centering
\includegraphics[width=0.5\textwidth]{./img/c4q6.png}
\end{figure}

\subsection*{Mais vetores}

\begin{questao}
\notaTutor{Faça questão que os estudantes comecem a resolver essa questão fazendo um esboço destes vetores. A posição precisa não é importante, mas aproximações coerentes com os ângulos (o segundo vetor deve estar no segundo quadrante) e normas (o primeiro deve ser mais longo que o segundo) dadas são.

Para o segundo item, não use o valor de seno e cosseno de 150\degree, mas sim o fato de o vetor formar um ângulo de 30\degree com o lado negativo do eixo $X$. Além disso, nesse item deve-se utilizar o valor tabelado para o cosseno e seno de 30\degree e não a calculadora.} Determine as coordenadas e represente em um plano cartesiano os vetores cuja norma e ângulo são dados abaixo.
\begin{enumerate}[a)]
\item Vetor $\overrightarrow{V}$ com norma $3$ e ângulo $40\degree$.
\item Vetor $\overrightarrow{U}$ com norma $2$ e ângulo $150\degree$.
\end{enumerate}
\end{questao}

\begin{gabarito}
	\begin{gabaritoQuestao}
		a) $(2,30;1,93)$, b) $(-\sqrt{3};1,00)$.
	\end{gabaritoQuestao}
\end{gabarito}

\subsection*{Área entre vetores}

\begin{questao}
Considere os dois vetores da questão 6 ($\overrightarrow{V}=(1,2)$ e $\overrightarrow{U}=(3,1)$). Vamos calcular a área do triângulo determinado pela origem e pela extremidade desses vetores.
\begin{enumerate}[a)]
\item Recupere da questão 6 os valores das normas e do ângulo entre os vetores e represente o triângulo em questão em um plano cartesiano com essas informações.
\item Trace a altura do triângulo relativa à $\overrightarrow{U}$.
\item Calcule o comprimento dessa altura usando o seno do ângulo entre os vetores.
\item Calcule a área do triângulo usando $\overrightarrow{U}$ como base.
\end{enumerate}
\end{questao}

\begin{gabarito}
	\begin{gabaritoQuestao}
		c) $\sqrt{10}/2$, d) $\frac{5}{2}$.
	\end{gabaritoQuestao}
\end{gabarito}

\begin{figure}[h]
\centering
\includegraphics[width=0.6\textwidth]{./img/c4q8.png}
\end{figure}

Se não tivéssemos usado valores numéricos para essa questão, você teria chegado à fórmula $A=\frac{1}{2} \cdot \text{lado 1} \times \text{lado 2} \times \sin(\alpha)$, em que o comprimento dos lados são iguais a norma dos vetores e $\alpha$ é o ângulo entre eles.

Imagine agora que o comprimento dos dois vetores estão fixos, mas é possível girar $\overrightarrow{A}$ em torno da origem. A medida que $\overrightarrow{A}$ se aproxima de $\overrightarrow{B}$, três variáveis se comportam de maneira semelhante: o ângulo $\alpha$ entre os vetores, a área do triângulo determinado pelos vetores e $\sin(\alpha)$ diminuem. Por outro lado, se afastarmos $\overrightarrow{A}$ de $\overrightarrow{B}$ até formar um ângulo reto essas três variáveis aumentam.

De certa maneira podemos interpretar essas três variáveis como sendo medidas de perpendicularidade entre dois vetores. Em Geometria Analítica essa ideia será capturada pela operação chamada \textbf{produto vetorial}.

\subsection*{Área entre vetores, 2}

\begin{questao}
\notaTutor{Essas duas questões tem o objetivo de mostrar a relação entre o seno e a área compreendida entre dois vetores. O mesmo raciocínio poderia ser aplicado ao paralelogramo determinado pelos dois vetores. Essa relação é central para a definição do produto vetorial. A grande diferença é que o produto vetorial generaliza essa relação para contextos com mais dimensões e foco nas coordenadas dos vetores, como seria de se esperar em geometria analítica.

Vale ressaltar que o ângulo entre os vetores também poderia ser calculado com auxílio de fórmulas para obter $\sin(a-b)$ e $\cos(a-b)$, mas esse não é o nosso foco já esse recurso é pouco utilizado em Geometria Analítica, onde as coordenadas dos vetores são utilizadas com mais frequência do que conceitos de geometria plana.} Calcule a área do triângulo determinado pelos vetores $\overrightarrow{V}=(2;2)$ e $\overrightarrow{U}=(4;1)$ usando a estratégia usada na questão anterior.
\end{questao}

\begin{gabarito}
	\begin{gabaritoQuestao}
		a) $2,42$.
	\end{gabaritoQuestao}
\end{gabarito}

Note que o valor obtido foi muito próximo de 3. Na realidade, se os cálculos fossem realizados sem aproximações, o resultado seria exatamente 3 e isso poderia ser verificado pelo método de ``contar quadradinhos'' já que as coordenadas dos vetores são números inteiros (bastaria considerar a área do retângulo $4 \times 2$ construído ao redor do triângulo e subtrair a área dos triângulos que não interessam. Tente!).

\subsection*{Projeção ortogonal}

\begin{questao}
Considere os dois vetores da questão anterior ($\overrightarrow{V}=(2,2)$ e $\overrightarrow{U}=(4,1)$).
\begin{enumerate}[a)]
\item Recupere da questão anterior os valores das normas e do ângulo entre os vetores e represente-os em um plano cartesiano.
\item Trace a altura do triângulo relativa à $\overrightarrow{U}$ e chame de $C$ a intersecção dessa altura com $\overrightarrow{U}$.
\item Calcule o comprimento do vetor $\overrightarrow{OC}$.
\end{enumerate}
\end{questao}

\begin{gabarito}
	\begin{gabaritoQuestao}
		a) $2.301$, b) $1.301$, c) $0.301$.
	\end{gabaritoQuestao}
\end{gabarito}

O vetor $\overrightarrow{OC}$ é chamado de \textbf{projeção ortogonal} de $\overrightarrow{V}$ em $\overrightarrow{U}$. A importância da projeção ortogonal está no fato de que ela pode ser interpretada como sendo a ``parte'' de $\overrightarrow{V}$ que atua na direção estabelecida por $\overrightarrow{U}$. Essa ideia é central em várias áreas da Física quando forças agem em um objeto e deseja-se estudar o efeito dessa força em uma determinada direção.

Note que neste caso, a projeção ortogonal está relacionada com o cosseno do ângulo entre os vetores. De modo análogo ao que concluímos para o seno/ área/ perpendicularidade, podemos dizer que o cosseno é uma medida de proximidade entre vetores (quando mais próximos, menor o ângulo e mais próximo de 1 o seu cosseno). Essa ideia será capturada e generalizada para outros contextos com mais de duas dimensões pela operação chamada \textbf{produto escalar}, que será bastante usada em Geometria Analítica.


\subsection*{Projeção ortogonal, 2}

\begin{reflita}
 Descreva com suas palavras como você deve proceder para obter o comprimento da projeção ortogonal de um vetor dado sobre um segundo vetor dado.
\end{reflita}

Use a resposta anterior para se orientar ao longo da resolução da próxima questão. Inclusive, se algo surgir durante a resolução, volte e reajuste a resposta à questão anterior.

\begin{questao}
\item Calcule o comprimento da projeção ortogonal de $\overrightarrow{V}=(1;\sqrt{3})$ em $\overrightarrow{U}=(2;2)$.
\end{questao}

\begin{gabarito}
	\begin{gabaritoQuestao}
		$1,93$.

\noindent\textbf{Questão 12:} a) $5$, b) $1$, c) dobrar também.
	\end{gabaritoQuestao}
\end{gabarito}


\subsection*{Encolhendo e esticando um vetor}

\begin{questao}
Considere o vetor $\overrightarrow{V}=(4;3)$.
\begin{enumerate}[a)]
\item Qual é a norma deste vetor?
\item Divida as duas dimensões de $\overrightarrow{V}$ pela sua norma de modo a obter um novo vetor que chamaremos de $\overrightarrow{V_1}$. Qual é a norma de $\overrightarrow{V_1}$?
\item Se você dobrar as dimensões de $\overrightarrow{V_1}$ o que vai ocorrer com a sua norma?
\end{enumerate}
\end{questao}

\begin{gabarito}
	\begin{gabaritoQuestao}
		a) $5$, b) $1$, c) dobrar também.
	\end{gabaritoQuestao}
\end{gabarito}

O que você fez no item b da questão anterior foi obter um \textbf{vetor unitário} (com norma igual a 1) que tem a mesma direção e sentido que um vetor dado. Para isso, você apenas dividiu as suas dimensões pela norma do vetor, ``encolhendo-o''. No item c você dobrou a norma desse vetor multiplicando as suas coordenadas por dois.

Esse processo será útil na seção Rumo ao livro-texto.

\section{Rumo ao livro-texto}

A questão a seguir não foi retirada de um dos livro-textos de Geometria Analítica, mas explora uma ideia bastante importante nessa disciplina (decomposição de vetores) em um contexto um pouco mais simples (duas dimensões) do que será feito na disciplina (três ou mais dimensões).

A proposta desta questão é reunir o que foi feito com seno, cosseno e norma nas questões anteriores para entender como decompor um vetor dado em dois vetores perpendiculares.

\begin{resolvida}
Decomponha o vetor $\overrightarrow{V}=(1;3)$ como soma de dois vetores perpendiculares de modo que um deles esteja na mesma direção do vetor $\overrightarrow{A}=(1;1)$.
\end{resolvida}

Vamos começar representando os vetores como mostrado abaixo.

\begin{figure}[h]
\centering
\includegraphics[width=0.4\textwidth]{./img/c4r1.jpg}
\end{figure}

A questão pede que decomponhamos $\overrightarrow{V}$ como soma de dois outros vetores. Vamos chamá-los de $\overrightarrow{V_A}$ e $\overrightarrow{V_B}$. Também é pedido que um deles, digamos $\overrightarrow{V_A}$, esteja na mesma direção de $\overrightarrow{A}$ enquanto que o outro, $\overrightarrow{V_B}$, seja perpendicular a ele.

Vamos focar em $\overrightarrow{V_A}$ por enquanto. Ele pode ser representado como mostrado abaixo.

\begin{figure}[h]
\centering
\includegraphics[width=0.4\textwidth]{./img/c4r2.jpg}
\end{figure}

Para obtê-lo, basta sabermos o comprimento da projeção ortogonal de $\overrightarrow{V}$ sobre $\overrightarrow{A}$ e então ajustarmos o comprimento de $\overrightarrow{A}$.

Como vimos em questões anteriores, o comprimento de $\overrightarrow{V_A}$ é obtido via cosseno do ângulo entre os vetores. O ângulo determinado por $\overrightarrow{A}$ é igual a 45\degree, pois suas duas dimensões são iguais. O cosseno do ângulo $\alpha$, determinado por $\overrightarrow{V}$, é igual a $\frac{1}{\sqrt{10}}=\frac{\sqrt{10}}{10} \approx 0,32$. Portanto, $\alpha = \arccos(0,32) \approx 71,6\degree$ e o ângulo entre os vetores é igual a $71,6-45=26,6$ graus.

Com base nessa informação, temos que o comprimento de $\overrightarrow{V_A}$ é dado por:

$$
\Arrowvert \overrightarrow{V_A} \Arrowvert = \cos(22,6) \cdot \sqrt{10} \approx 0,89 \cdot \sqrt{10}
$$

Agora, como $\overrightarrow{V_A}$ aponta para a mesma direção de $\overrightarrow{A}$, vamos fazer com que a norma de $\overrightarrow{A}$ seja igual ao valor que foi obtido acima.

Primeiro, dividimos as dimensões de $\overrightarrow{V_A}$ pela sua norma ($\sqrt{2}$): $(\frac{1}{\sqrt{2}};\frac{1}{\sqrt{2}})$.

Segundo, multiplicamos o resultado obtido acima pela norma desejada ($0,89 \cdot \sqrt{10}$): $(\frac{0,89 \cdot \sqrt{10}}{\sqrt{2}};\frac{0,89 \cdot \sqrt{10}}{\sqrt{2}}) = (0,89 \cdot \sqrt{5};0,89 \cdot \sqrt{5})$.

Esse é o vetor $\overrightarrow{V_A}$. Note que suas coordenadas podem ser approximadas para $(2,0;2,0)$.

A segunda parte da questão é mais simples. O enunciado nos pediu que o $\overrightarrow{V}$ seja decomposto como uma soma de dois vetores, ou seja, $\overrightarrow{V}=\overrightarrow{V_A}+\overrightarrow{V_B}$, como já obtivemos $\overrightarrow{V_A}$, temos que $\overrightarrow{V_B}=\overrightarrow{V}-\overrightarrow{V_A}$, logo: $\overrightarrow{V_B}=(1;3)-(2;2)=(-1;1)$.

Representando todos os vetores na figura abaixo, note que as dimensões de $\overrightarrow{V_B}$ são compatíveis com o que foi pedido pela questão, $\overrightarrow{V_A}$ e $\overrightarrow{V_B}$ serem perpendiculares, e eles são a resposta final.

\begin{figure}[h]
\centering
\includegraphics[width=0.4\textwidth]{./img/c4r3.jpg}
\end{figure}

Agora, tente aplicar o mesmo raciocínio para resolver a questão a seguir. Tente ser estratégico no uso de frações ou aproximações.

\begin{resolva}
Decomponha o vetor $\overrightarrow{V}=(3;4)$ como soma de dois vetores perpendiculares de modo que um deles esteja na mesma direção do vetor $\overrightarrow{A}=(12;5)$.
\end{resolva}

%\end{comment}
\paraTutores

\subsection{Questões adicionais}

\begin{adicional}
\notaTutor{Essa questão é similar ao que foi discutido na seção Rumo ao Livro-texto, mas usa pela primeira vez um vetor que não está no primeiro quadrante. Isso foi usado com reserva ao longo deste capítulo porque não estamos trabalhando com ângulos maiores do que 90\degree por enquanto. Entretanto, essa questão pode ser resolvida sem que isso seja necessário, basta observar o comportamento dos ângulos após representar os vetores no plano cartesiano.} Decomponha o vetor $\overrightarrow{V}=(2;2\sqrt{3})$ em uma soma de vetores perpendiculares de modo que um deles esteja na direção sugerida pelo vetor $\overrightarrow{A}=(4;-2)$.
\end{adicional}



\begin{adicional}
\notaTutor{O objetivo dessa questão é que os estudantes discutam como deve se comportar a decomposição de vetores cujo ângulo de separação seja maior do que 90\degree. O cálculo em si não é importante, pois o procedimento já foi repetido pelos estudantes algumas vezes ao longo do capítulo, por isso incentive: o cuidado com o esboço do item a e a discussão em torno do item b.} Considere os vetores $\overrightarrow{A}=(\sqrt{3};1)$ e $\overrightarrow{B}=(-1;1)$.
\begin{enumerate}[a)]
\item Esboce com cuidado ambos em um plano cartesiano.
\item Esboce a decomposição de $\overrightarrow{B}$ como uma soma de vetores perpendiculares de modo que um deles esteja na direção sugerida pelo vetor $\overrightarrow{A}$
\item Estime visualmente as coordenadas da decomposição de $\overrightarrow{B}$.
\end{enumerate}
\end{adicional}


Como este capítulo foi totalente baseado em vetores com apenas duas dimensões, não foi possível indicar materiais complementares. Caso algum estudante chegue a este ponto antes do final das atividades, sugerimos a leitura da seção 8.1 do livro Álgebra Linear, ed José Luiz Boldrini e colegas, para uma introdução ao produto escalar e vetorial ou da seção 3.1 do livro \href{https://regijs.github.io/livros.html}{Matrizes, Vetores e Geometria Analítica} sobre soma de vetores, subtração e multiplicação por escalar.

\newpage

\section{Gabarito}

\imprimeGabarito

%\end{comment}
\paraAlunos

\section{Auto-avaliação final}
Avalie o quanto você acha que sabe sobre os seguintes itens após ter resolvido as questões deste capítulo.

\begin{center}
 \begin{tabular}{|p{25mm}||p{10mm}|p{10mm}|p{10mm}|p{10mm}|} 
 \hline
   & Nada & Muito pouco & Noções gerais & Muito\\
 \hline
 Calcular seno e cosseno em um triângulo retângulo &  &  &  &  \\ 
 \hline
 Obter um ângulo sabendo o seu seno ou cosseno &  &  &  &  \\
 \hline
 Usar seno e cosseno em uma calculadora &  &  &  &  \\
 \hline
 Representar vetores &  &  &  &  \\
 \hline
\end{tabular}
\end{center}

Cheque como foi o seu progresso comparando essas respostas com as que você deu antes de estudar este capítulo. Caso você não tenha atingido o nível ``Bastante''  em algum dos tópicos acima, liste abaixo qual ação concreta você fará nos próximos dias para atingi-lo:

%\end{comment}
\paraAmbos

\end{document}
