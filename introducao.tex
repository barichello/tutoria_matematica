\begin{document}

\chapter{Introdução}

%\end{comment}
\paraTutores

Este material foi desenvolvido para estudantes ingressantes de cursos de Exatas como um suporte para o conteúdo das disciplinas de Cálculo Diferencial e Integral e Geometria Analítica.

Em termos de conteúdo, não queremos reforçar os tópicos dessas duas disciplinas nem promover uma revisão de todo o Ensino Médio. Nossa intenção é revisitar alguns tópicos do Ensino Médio enfatizando aspectos que estejam diretamente relacionados com tópicos específicos de Cálculo Diferencial e Integral e Geometria Analítica. Ou seja, queremos relacionar aquilo que você aprendeu no Ensino Médio com aquilo que terá que aprender em breve nessas duas disciplinas

Em termos da abordagem, nosso foco não é em fluência na resolução de exercícios, mas no entendimento mais conceitual dos tópicos em questão. O que buscamos aqui é criar um  momento de estudo, no qual o trabalho é feito pelo leitor com eventual ajuda de algum tutor.

\section{O que esperamos de você, tutor}

\subsection{Conheça o material}

Esperamos que você, além de resolver todas as questões propostas aos estudantes, leia atentamente os comentários acerca de cada questão para que esteja ciente de algumas nuances que podem passar despercebidas. A ordem das questões, os itens de cada uma delas e até mesmo os valores numéricos de cada item foram pensados cuidadosamente para que o estudante tenha uma experiência gradual e novos elementos sejam inseridos apenas quando os anteriores já tenham sido devidamente abordados.

Sugestões de onde podem surgir dificuldades e como elas podem ser abordadas, bem como de exemplos adicionais e variações das questões, estão presentes ao longo de todo este material. Por isso, esperamos que ele seja uma leitura útil para antes dos encontro e uma referência para durante cada um deles.

\subsection{Evite usar lousa}

Pode parecer uma sugestão estranha, mas a mensagem que queremos passar com ela é que a atividade de tutoria não deve virar uma aula expositiva sobre o conteúdo, muito menos aula de resolução de exercícios em que o tutor resolve os exercícios e os estudantes anotam a resolução. Experiências similares em outras universidades mostram que o envolvimento ativo dos estudantes na resolução das questões é fundamental para o sucesso dessa proposta.

Uma estória exagerada pode ser útil para explicar o que esperamos com essa recomendação: um professor universitário, durante seu horário de atendimento, costumava resolver questões em folhas de rascunhos para os estudantes que vinham procurá-lo e, ao final da resolução, perguntava ``Você entendeu?'' e se a resposta fosse sim, ele jogava o papel fora e dizia ``ótimo, então você pode fazer sozinho agora''.

Obviamente, você não precisa agir dessa maneira. Confiamos na sua sensibilidade para decidir o que é melhor para a sua turma de estudantes, mas a mensagem que queremos passar é de que durante as atividades da tutoria, são os estudantes que devem fazer o trabalho, você está ali apenas para oferecer suporte.

\subsection{Responda com perguntas}

Esta atitude está profundamente ligada à sugestão anterior.

Sempre que possível tente responder as perguntas feitas pelos estudantes com novas perguntas que sugiram caminhos ao invés de dar a solução. ``O que exatamente você já tentou fazer?'', ``Você já tentou isso?'', ``Você já checou como fulano resolveu''?, ``Você leu o texto antes da questão?'', ``Há algum termo específico que você não conhece?'', são algumas das questões que podem ser usadas em praticamente qualquer ponto dos cadernos. A leitura atenta do material do tutor deve lhe ajudar na identificação e escolha de boas perguntas.

Eventualmente alguns pontos precisam ser explicados de forma mais expositiva, como definições ou propriedades que não sejam discutidas no material. Quando esse for o caso, tente evitar resolver a questão específica que gerou a dúvida e, se o fizer, proponha uma nova questão ao tutorado. 

\subsection{Use o grupo}

Duas das intervenções com maior impacto em termos de aprendizagem são tutoria por colegas e apoio individualizado. Você pode atingir esse efeito no seu grupo de estudantes pedindo que eles se ajudem sempre que houver dúvidas em questões que já tenham sido resolvidas por outros estudantes.

Inclusive, você vai notar que algumas questões pedem interação entre os estudantes explicitamente, mas você pode expandir isso sempre que parecer adequado. O fato de a quantidade de questões no material não ser muito grande tem como objetivo justamente viabilizar esse tipo de ação, a qual pode parecer lenta inicialmente mas tem grande potencial de impacto em termos de aprendizagem.

\subsection{Não tenha pressa}

O objetivo das atividades de tutoria não é promover a fluência com certos procedimentos, mas promover uma conexão significativa dos conteúdos vistos anteriormente com o que está sendo discutido nas disciplinas principais. Portanto, não apresse os estudantes para que terminem os capítulos dentro de certos prazos. É preferível que um estudante não resolva todas as questões tendo compreendido bem as que resolveu do que tenha obtido a resposta correta em todas através de um engajamento superficial.

\subsection{Planeje}

O material que você tem em mãos foi concebido de modo a poder ser utilizado ao longo do semestre, paralelamente às disciplinas, exigindo cerca de 3 horas de estudo por semana.

No total, há 11 capítulos que devem ser estudados em ordem. Os 7 primeiros podem ser estudandos antes mesmo do início das aulas. Idealmente, eles devem ser estudados antes que o professor das disciplinas principais abordem os tópicos relacionados nas suas aulas. Já os 4 últimos capítulos devem ser estudados assim que o professor de cálculo abordados os tópicos na disciplina regular.

\section{Estrutura do material dos estudantes}

O material dos estudantes está estruturado da seguinte maneira:

\begin{enumerate}
 \item Apresentação
 \item Pré-requisitos e Auto-avaliação inicial: explicando quais são os pré-requisitos do capítulo, que eventualmente precisam ser estudados antes dos encontros, e uma auto-avaliação sobre o quanto eles acreditam que sabem sobre alguns tópicos chave para o capítulo;
 \item Avaliação Diagnóstica: deve ser resolvida ao final do último encontro do capítulo anterior e te orientará sobre em qual ponto cada estudante deve começar;
 \item Questões: onde se concentra a maior parte do conteúdo, formado por questões e por textos discutindo os tópicos em pauta. É importante que os textos sejam lidos pelos estudantes, pois ali são feitas várias conexões fundamentais;
 \item Rumo ao livro texto: uma seção com o objetivo de propor questões ou leituras que explicitamente conectem o trabalho deste material com os livros-texto das disciplinas oficiais;
 \item Gabarito
 \item Registro de progresso: deve ser preenchido pelos estudantes ao final do período alocado para cada capítulo, fotografado por você e enviado ao professor coordenador;
 \item Auto-avaliação final: oferecendo uma oportunidade para o estudante comparar a sua evolução e traçar metas de estudo.
\end{enumerate}

O seu material segue estrutura parecida, mas com alguns adicionais.

\begin{enumerate}
 \item O Quadro de orientação te ajuda a decidir em que ponto os estudantes devem começar cada capítulo de acordo com o desempenho na Avaliação Diagnóstica. Você pode decidir não segui-lo por conta de especificidades dos estudantes e grupos, mas leve-o em conta antes de tomar essa decisão;
 \item Ao longo das Questões, há notas laterais salientando aspectos importantes, erros esperados ou estratégias de intervenção para certas situações. É fundamental que essas sugestões sejam lidas e as questões resolvidas por você antes dos encontros;
 \item Questões Adicionais traz algumas questões que podem ser propostas aos estudantes que completarem o capítulo com muita antecedência. Use-as com moderação, pois a intenção não é que esse material represente mais um fardo na rotina de estudos dos ingressantes.
\end{enumerate}

Folheie o seu material para se familiarizar com ele e em caso de dúvidas, procure o professor coordenador.

\section{A rotina ideal}

A lista abaixo descreve a rotina que tínhamos em mente quando criamos a estrutura do material. Obviamente, você deve adaptá-la às suas necessidade, mas sugerimos tentar incorporar essa estrutura o máximo possível.

\begin{enumerate}
 \item Todo capítulo é iniciado com uma pequena Avaliação Diagnóstica. Essa avaliação deve ser resolvida pelos seus estudantes no final do encontro anterior, quando o capítulo anterior for finalizado;
 \item Resolva as questões do próximo capítulo e leia as orientações para o tutor;
 \item Corrija as questões da Avaliação Diagnóstica e anote em que ponto do capítulo seguinte cada estudante deverá começar;
 \item Durante as 3 horas de atividades, que podem estar distribuídas em 2 ou 3 encontros, os estudantes deverão resolver as questões do caderno;
 \item Ao final do último encontro lembre-se de pedir que resolvam a Avaliação Diagnóstica para o próximo capítulo e de preencherem a auto-avaliação.
\end{enumerate}

%\end{comment}
\paraAlunos

\section{Apresentação}

Este material foi desenvolvido para ajudar estudantes ingressantes de cursos de Exatas em tópicos cruciais para o conteúdo das disciplinas de Cálculo Diferencial e Integral e Geometria Analítica.

Temos três grandes objetivos com esse material. Primeiro, reforçar alguns tópicos matemáticos importantes do Ensino Médio. Segundo, reforçar a conexão entre o que você aprendeu nos últimos anos com o que será ensinado nessas disciplinas. Terceiro, criar um material que permita ao estudante estudar ativamente, possivelmente com suporte de um tutor.

O material foi desenvolvido tendo em mente 3 horas de estudo por semana durante as quais você deverá resolver as atividades propostas no seu material.

\section{O que esperamos de você}

\textbf{Vá com calma}. A quantidade de atividades propostas foi pensada para que haja tempo para resolver todas as questões durante os encontros. Portanto, não corra. Leia atentamente as questões e os textos explicativos antes e depois delas. Uma grande parte da aprendizagem esperada vem dessas explicações. Caso você não termine algum capítulo, não se preocupe. É mais importante que você compreenda cada tópico visto do que chegue ao final de tudo apressadamente.
 
\textbf{Seja ativo}. Ao ler o material, tenha certeza de que você entendeu o conteúdo. Volte e cheque as referências, refaça cálculos se for necessário e faça anotações. Jamais deixe de registrar o processo de resolução de uma questão de modo que você consiga relê-lo no futuro se desejar. Recomendamos que você faça anotações tanto no texto quanto nas suas resoluções que lhe permitam tanto entender  melhor o que foi feito quanto re-entender caso um dia você o consulte novamente. 

\textbf{Pergunte}. Peça ajuda aos colegas e ao seu tutor caso não tenha conseguido entender alguma coisa. Não se sinta inibido, pois outros colegas podem ter as mesmas dúvidas que você ou serem capazes de ter explicar algo a partir de um ponto muito parecido com o que você está agora. Mas, ao invés de respostas prontas, procure sugestões ou esclarecimentos que lhe permitam resolver as questões e entender os conceitos de maneira independente.

\textbf{Mantenha o engajamento.} Se você não conseguir terminar algum capítulo, não se preocupe. Este material foi concebido pensando nessa possibilidade: todo trabalho feito aqui deve te ajudar nas disciplinas principais cedo ou tarde. Por outro lado, se em algum momento o conteúdo parecer inútil ou muito fácil, mantenha o engajamento pois os tópicos foram cuidadosamente escolhidos e vocẽ notará o efeito do material em breve.

\textbf{Registre o seu progresso}. Não deixe de registrar o seu progresso ao final de cada capítulo e de preencher a auto-avaliação. Isso é importante para um bom acompanhamento das atividades (tanto por você mesmo quanto por tutores, se houver).

\textbf{Reflita}. Haverão questões explicitamente focadas em lhe fazer refletir sobre os tópicos discutidos e oportunidades para que você pense sobre o que você sabe, o quanto aprendeu e o que pode fazer para melhorar. Essas habilidades são importantes, não as menospreze! 

\textbf{Não negligencie as disciplinas principais}. O objetivo da tutoria é te ajudar com as duas disciplinas principais e não ser mais um disciplinas por si só. Use o tempo da tutoria para a tutoria, mas não retire tempo de estudo das principais para investir na tutoria. Se a carga de trabalho estiver demais, conversa com seu tutor.

\section{Os capítulos}

Este é composto por 10 capítulos, divididos em 2 grandes grupos.

O primeiro grupo, com os 7 primeiros capítulos, cobrem conteúdo do Ensino Médio que estão diretamente ligados a tópicos de Cálculo Diferencial e Geometria Analítica. Eles foram concebidos para serem estudados ao longo do semestre, paralelamente às disciplinas e, sempre que possível, antes de os professores delas abordarem os tópicos de cada um.

O segundo grupo, com os 3 capítulos finais, cobrem tópicos bastante específicos de Cálculo Diferencial e Integral. Eles foram concebidos para serem estudados logo que os tópicos foram ensinados pelo professor da disciplina.

\section{Estrutura do material}

A maioria dos capítulos deste material, a partir do próximo, segue a mesma estrutura:

\begin{enumerate}
 \item Apresentação: explicando porque o tópico em questão foi escolhido para o material e em que ele deve te ajudar nas disciplinas de Cálculo Diferencial e Integral e Geometria Analítica;
 \item Pré-requisitos e Auto-avaliação inicial: explicando quais são os pré-requisitos do capítulo, que eventualmente precisam ser estudados antes dos encontros, e oferecendo uma oportunidade para você refletir sobre o seu conhecimento em Matemática;
 \item Avaliação Diagnóstica: para que você inicie as atividades do capítulo em um ponto compatível com o seu conhecimento. Dê o seu melhor e a resolva sozinho. Essa avaliação deve ser resolvida ao final do último encontro do capítulo anterior. Assim, você pode aproveitar os dias antes do próximo encontro para estudar os pré-requisitos e os tópicos que foram difíceis pra você na Avaliação Diagnóstica;
 \item Questões: onde se concentra a maior parte do conteúdo, formado por questões e por texto discutindo os tópicos em pauta;
 \item Rumo ao livro texto: com o objetivo de propor questões ou leituras que explicitamente conectem o trabalho que você acabou de fazer com o livros-texto das disciplinas oficiais;
 \item Gabarito: contém as respostas para quase todas as questões. Use para checar as suas respostas quando você terminar de resolver uma questão, não para copiar a resposta final ou para ``forçar'' o caminho da resolução;
 \item Registro de progresso: para que você registre quais questões resolveu (não importa se certo ou errado). Essa seção é importante para que possamos acompanhar a implementação do projeto e aprimorar o material;
 \item Auto-avaliação final: oferecendo uma oportunidade para você comparar a sua evolução e traçar metas de estudo.
\end{enumerate}

\section{Referências essenciais}

Este material é bastante auto-contido, mas alguns outros livros serão referenciados tanto para sugerir leituras que expliquem tópicos não cobertos pelo material quanto para indicar leituras de aprofundamento ou continuidade.

A lista a seguir contém todas as referências que serão usadas ao longo do material. Sugerimos que você tenha esses materiais disponíveis durante as atividades da tutoria. Todos podem ser encontrados na biblioteca ou na internet.

\begin{itemize}
 \item O livro digital \sugestao{Matemática Básica volume 1}, de Francisco Magalhães Gomes, professor do IMECC. Disponível em http://www.ime.unicamp.br/~chico
 \item O livro digital \sugestao{Matrizes, Vetores e Geometria Analítica}, de Reginaldo J. Santos. Disponível em www.mat.ufmg.br/~regi
 \item O livro físico \sugestao{Álgebra Linear}, de José Luiz Boldrini e outros. Amplamente disponível na biblioteca.
\end{itemize}

Existem também diversos portais com vídeos abordando tópicos de matemática na internet. Enquanto vários deles são bons, alguns não são. A sugestão que fazemos é o Portal do Saber (portaldosaber.obmep.org.br). Ele se destaca pela organização, qualidade dos vídeos, recursos disponíveis e uniformidade do material.

\end{document}
